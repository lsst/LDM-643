
\section{Introduction}

As LSST DM moves from construction through commissioning and into operations 
a number of rehearsals have been proposed to help prepare for the execution 
of the survey.  Specific rehearsals are outlined in \citedsp{LDM-503} but in 
the more comprehensive cases, i.e. the Operations (Ops) Rehearsals (LDM 503-09, LDM 503-11, and 
LDM 503-12), the contents of those document alone, do not sufficiently outline 
the scope, content, action and interaction that are being rehearsed.  
From the software side, \citedsp{LDM-564}
summarizes the DM software features that should be available and helpfully 
identifies those software releases in the context of the rehearsals.  
However, the Ops Rehearsals are not simply periods to test hardware and 
software systems, they are opportunities to develop and understand operations 
processes and to observe the interactions of hardware, software and personnel.

This document attempts to outline the Ops Rehearsals in greater detail for the following reasons:
\begin{itemize}
\item Depending on the impact, missing or late software features and hardware systems may either
require mitigation (e.g., shims, fake data, etc...) or might even be grounds for 
postponement.
\item The purpose of an Ops Rehearsal is not to debug 
a freshly deployed system, but rather to understand whether that system does
what is needed.
\item The effort to carry out rehearsals will 
require coordination of personnel and facilities.
\end{itemize}

Note: This remains a work in progress.  The current draft attempts to describe the 
process for the first Ops rehearsal.  The subsequent rehearsals have only been 
outline to the extent to properly understand their scope.

\clearpage

\input rehearsal_1.tex

\input rehearsal_2.tex

\input rehearsal_3.tex

