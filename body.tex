
\section{Introduction}

As LSST DM moves from construction through commissioning and into operations 
a number of rehearsals have been proposed to help prepare for the execution 
of the survey.  Specific rehearsals are outlined in \citedsp{LDM-503} but in 
the more comprehensive cases, i.e. the Operations (Ops) Rehearsals (LDM 503-09, LDM 503-11, and 
LDM 503-12), the contents of those document alone, do not sufficiently outline 
the scope, content, action and interaction that are being rehearsed.  
From the software side, \citedsp{LDM-564}
summarizes the DM software features that should be available and helpfully 
identifies those software releases in the context of the rehearsals.  
However, the Ops Rehearsals are not simply periods to test hardware and 
software systems, they are opportunities to develop and understand operations 
processes and to observe the interactions of hardware, software and personnel.

This document attempts to outline the Ops Rehearsals in greater detail for the following reasons:
\begin{itemize}
\item depending on the impact, missing or late software features and hardware systems may either
require mitigation (e.g., shims, fake data, etc...) or might even be grounds for 
postponement
\item the purpose of an Ops Rehearsal is not to debug 
a freshly deployed system, but rather to understand whether that system does
what is needed
\item the effort to carry out rehearsals will 
require coordination of personnel and facilities
\end{itemize}

Note: The current draft attempts to describe the process for the first Ops 
rehearsal in an attempt to illuminate the process for follow on rehearsals.


\section{LDM 503-09: Operations Rehearsal \#1 for Commissioning}

\underline{Nominal Date:} April 2019

\underline{Original Description:}
\begin{itemize}
\item Choose TBD weeks during commissioning.
\item Pick which parts of plan we could 
rehearse.
\item The commissioning team (via Chuck Claver) suggests Instrument Signal Removal should be the focus 
of this (or the next rehearsal)
\end{itemize}

\clearpage

\subsection{An Updated Goal:}

The primary goal is to simulate nominal operations, both daytime and nighttime, for a 2--3 day 
period including the daily meeting(s) that would occur among the SciOps and 
Data Facility staff.  These activities will be accompanied by simulated 
observations obtained in a ``sampling'' mode in order to exercise:
\begin{enumerate}
\item the transfer, archiving and ingestion of raw data
\item offline processing of 
calibrations and science data
\item curation of the resulting data products
\end{enumerate}

At the time of this initial rehearsal, we do not expect a functioning 
observatory system, instead:

\begin{itemize}[topsep=0pt]
\item Sampling mode has been used to describe early LSST commissioning 
observations where observations occur based on the needs of the commissioning 
team.  Such observations would typically include some basic set of calibrations
(e.g. biases and flats) followed by nighttime observations that might be used to 
test and quantify system performance.

\item A basic set of raw data will be transferred to a mountaintop computer 
which will then in turn mimic observations by sending those images from the
summit to NCSA via the long-haul networks.

\item The contents of the dataset will be minimal (no larger than the 
calibration and nightly observations that might be expected from ComCam).  
The current plan is to draw from a suitable set of test-stand data and 
then present these as though they were coming from the telescope.

\item On arrival at the LDF the observations will be ingested into the current 
data-backbone which can in turn be used to feed the data through a batch 
production service to produce ``calibrations'' and ``reduced science products.''

\item In the context of this rehearsal, the sophistication (or correctness) 
of the pipeline(s) is not paramount.  What is important is that the raw and 
resulting products are tracked and can be superficially examined by LDF and 
SciOps team members.  The degree of realism would depend on both the data
being sent and availability of working pipeline tasks.
\end{itemize}


\clearpage
\subsection{Pre-Requisites:}

There are three broad categories of pre-requisites that are needed:
\begin{enumerate}
\item Persons must be identified to fill roles within the rehearsals.
\item Services (or facsimiles) need to exist that will be used/tested throughout the 
rehearsal.
\item Elements that would not otherwise 
be available in the pre-operations LSST project will be prepared.
\end{enumerate}
 
\subsubsection{Pre-Requisites: Roles:}

Persons need to be identified that will staff various roles in this 
rehearsal.  These roles are drawn from those in operations which come
from three groups: Observatory Operations (ObsOps), Science Operations 
(SciOps), and the LSST Data Facility (LDF).
\begin{itemize}[topsep=-8pt]
\item Coordinator:  This person acts as an independent executor of the 
rehearsal.  They would be responsible for executing outside actions that drive the 
simulation (e.g., initiating a script that would start data flowing from 
summit to LDF).
\item ObsOps, Observing Specialist
\item ObsOps, ???
\item ObsOps, ???
\item SciOps, QA Scientist
\item SciOps, Verification and Validation Scientist
\item LDF, Operator
\item LDF, QA Scientist
\item LDF, Admin
\end{itemize}

\subsubsection{Pre-Requisites: Services \& Service Components:}

The broader pieces of the DM system that need to operate for this rehearsal are:
\begin{itemize}[topsep=-8pt]
\item A service must operate at the mountaintop that will send data.  This can 
be as simple as a shell script that draws from a list of files and transfers 
them to NCSA with some cadence. 
\item The long-haul networks need to be available at the time of this rehearsal.
\item A data backbone endpoint to receive and ingest incoming files must exist.
\item A mechanism must exist to distribute jobs to a compute resource 
to process the "new" data--Batch Production.
\item A workflow system to configure and launch jobs must exist.
\item Pipeline(s) to processes the data must be in place.
\item A minimally functional science platform where raw and processed products can be 
examined by staff must exist.
\end{itemize}

Additional monitoring services for the networks, batch production, compute 
resource(s), and data backbone are desirable.

\subsubsection{Pre-Requisites: Work:}\label{prework}

Work that must be completed prior to Ops Rehearsal but which is not part of 
the DM development:
\begin{itemize}[topsep=-8pt]
\item Generate a mock data set.  This must have the ability to be ingested with 
either Gen2 or Gen3 Butler.  It is not necessary that the generated data 
products be curated for a long period.
\item Create shim service that sends data from summit to LDF.
\item Specify appropriate pipeline(s) that will be run during the rehearsal.
\item Test that services in the preceding section can adequately function 
for the purposes of this rehearsal.
\item Allocate compute and storage resources and specify location of stored 
products.
\item Location to record information about incidents, problem backlog, 
processing and QA summaries (for the initial test this could be as simple a 
set of confluence pages).
\end{itemize}


\subsection{Rehearsal Outline:}
During normal operations the time observing occurs depends on local nighttime
in Chile.  This is not necessary for the rehearsal so that data delivery and
can be shifted to occur in a normal working day.  Prior to the execution of 
the rehearsal the work outlined in Section~\ref{prework} must be completed
and tested.
%%\item Pre-checklist: Assemble proto-ops team, all component services 
%%from DM are ready with payloads, data sets, configurations, etc. 
%%(assumes pre-integration work).

A basic outline of the processes that would occur during for this rehearsal 
follows:
\begin{itemize}[topsep=-8pt]
\item (ALL: ObsOps+SciOps+LDF) Afternoon stand-up operations meeting.
\item (ObsOps) Mock transmit nightly cals to LDF for ingestion.
\item (LDF) Generate nightly calibrations.
\item (SciOps) Select configuration and calibrations.
\item (ObsOps) Mock transmit nightly science images and ingest
\item (LDF) Run science pipeline (.e.g. ISR) in offline/batch mode.
\item (LDF) Generate feedback on processing for discussion in stand-ups.
\item (SciOps) Examine input and output data from nightly observations and
processing.
\item (SciOps) Generate feedback for discussion in stand-ups.
\item (ALL) Monitor progress of nightly “campaigns,” characterize and assess, 
make records of failures, diagnose issues, generate problem backlog.
\item (ALL) Create mock nightly reports.
\item Repeat (a total of 3 times)
\end{itemize}

While there are "realistic" components within the outline, much of focus by 
the actors should be on the processes.  How is this going to look in 
day-to-day operations?  If there are problems, what happens?  Who gets a 
call and when?  What information needs to be available between a geographically
distributed team? (and when?)  Are the lines of communication between those
groups sufficient?

%\clearpage

%\subsection{Software Products and Services Needed:}
%
%Based on the actions being undertaken in this rehearsal the following services are needed: 
%a shim for Camera DAQ and Archiving Services, Data Backbone Services (w/ minimal ability to make data visible to LSP), 
%and Batch Production Services with appropriate Pipelines.   
%These services are implemented within the following sofware products:
%\begin{itemize}
%\item Batch Production Software (Michelle Butler)
%\item Science Pipeline Software (Robert Lupton)
%\item Supporting Software (e.g., Data Butler): (Jim Bosch)
%\end{itemize}
%Since many of these are in a nascent state, often a shim (or some user-driven actions) 
%may be needed to emulate some elements.
%
%\clearpage

\subsection{Assess:}

Among the activities in the Rehearsal Outline it is expected that some might influence the long-term development
within DM.  For example exercising tools/services (e.g. the LSP) with a mind toward operational needs.  Thought
toward the processes and metrics needed to make decisions about configuration, calibration selection, 
and production success/failure. 

\begin{itemize}
\item Was the rehearsal successful? How long did it take? What anomalies/failure modes were identified, and how did team cope? 
\item What fixes are needed, and on what timescale (e.g., next ops rehearsal, or we are go for commissioning)? 
\item What improvements in procedures, documentation, frameworks, systems, and algorithms were identified?
\item Budget time and effort to plan and execute priority changes and improvements, and plan for next rehearsal.
\end{itemize}


%
%
%

\section{LDM 503-11: Operations Rehearsal \#2 for Commissioning}

\underline{Original Description:}
More complete commissioning rehearsal: how do the scientists look at data? 
How do they provide feedback to the telescope? How do we create/update 
calibrations?

%
%
%

\section{LDM 503-12: Operations Rehearsal \#3 for Commissioning}

\underline{Original Description:}
Dress rehearsal: commissioning starts in April so by this stage we should 
be ready to do everything needed.


