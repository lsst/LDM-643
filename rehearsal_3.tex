
\section{LDM 503-12: Operations Rehearsal \#3 for Commissioning}

\underline{Nominal Date:} August 2021

\underline{Original Description:}\\
Dress rehearsal: commissioning starts in April so by this stage we should 
be ready to do everything needed.

\subsection{An Updated Goal:}

Here the primary goal is to rehearse for commissioning operations prior to 
LSSTCam start of integration and test (i.e. while LSSTCam is on the summit
but not yet integrated on the telescope).  Similar to Ops Rehearsal \#2, 
we would emulate both daytime and nighttime, 
for a 3--5 days, would include daily meetings, exercise data movement and
processing.  Additionally this rehearsal could include: application of software 
changes, simulated problems, or non-standard (unprocessable) engineering 
observations.  

\begin{itemize}[topsep=-8pt]
\item LSSTCam should be at the summit in the clean room on its test stand.
LSSTCam would be exercised with its Camera Control System to obtain test-stand
images and send them through the DAQ for archiving and batch processing.  This
could be supplemented with on-sky data from ComCam to exercise pipeline 
processing.

\item The contents of the data would roughly match those expected during 
LSSTCam verification activities but the use of on-sky data from ComCam would
not be supplemented (to ``simulate" data volume) but real-time processing
could be exercised.

\item On arrival at the LDF the observations will be ingested into the 
data-backbone which can in turn be used to feed the data through a batch
production service to produce calibrations, reduced science products, and
quality assessments.

\item Similar to the other Ops Rehearsal \#1, the sophistication (or correctness)
of the pipelines are not paramount.  What is important is that the raw and
resulting data products are tracked and can be examined by LDF and
SciOps team members.  The degree of realism would depend on both the data
being sent and availability of working pipeline tasks.
\end{itemize}

\clearpage


