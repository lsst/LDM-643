
\section{LDM 503-11: Operations Rehearsal \#2 for Commissioning}

\underline{Nominal Date:} December 2019 - February 2020

\underline{Original Description:}\\
More complete commissioning rehearsal: 
\begin{itemize}[topsep=-8pt]
\item How do the scientists look at data? 
\item How do they provide feedback to the telescope?
\item How do we create calibrations?
\item How do we update calibrations?
\end{itemize}

\subsection{An Updated Goal:}

The primary goal is to rehearse for commissioning operations prior to the ComCam 
verification and validation era (including the mini-surveys).  
Similar to Ops Rehearsal \#1, we would emulate both daytime and nighttime, 
for a 3--5 days, would include daily meetings, exercise data movement and
processing.  Additionally this rehearsal could include: application of software 
changes, simulated outages, or non-standard (unprocessable) engineering 
observations.  If the Auxiliary Telescope Spectrograph has become available,
one alternative or extension that should be considered would be to use AuxTel 
data and a pipeline as part of these exercises.

In the current time frame of this rehearsal, we do not expect a functioning 
telescope + camera.  Instead:

\begin{itemize}[topsep=-8pt]
\item Will use AuxTel/LATISS to obtain 'real' imaging data.  We will attempt
to exercise as many elements of the emergent automatic data transfer as possible 
(including data transfer to base) but at a mimimum to the LDF.

\item The contents of the dataset will roughly match those expected during 
ComCam verification activities (except that they will be at a scale of LATISS
rather than ComCam) Thus, the dataset would be comprised of calibration and 
nightly observations but could also include engineering data (that might not 
be processed with a normal pipeline).

\item On arrival at the LDF the observations will be ingested into the current
data-backbone which can in turn be used to feed the data through a batch
production service to produce ``calibrations'' and ``reduced science products.''
If the DAQ2.5 hardware/software are available then early prompt processing could
be attempted (meaning processing that would be triggered by arrival of new observations).

\item Similar to the Ops Rehearsal \#1, the sophistication (or correctness)
of the pipelines are not paramount.  What is important is that the raw and
resulting data products are tracked and can be superficially examined by LDF and
SciOps team members.  The degree of realism would depend on both the data
being sent and availability of working pipeline tasks.

\item Once ComCam is available on the test stand at the summit, a follow-on 
rehearsal (nominally OPS Rehearsal 2b) should be considered to further exercise
capabilities at raft-scale and better prepare for ComCam during commissioning.
Processing would have to focus on exercising Calibration Products Pipelines (CPP) 
but could also focus on fast turn around through use of the OODS and Commissioning
Cluster at the Base Center.
\end{itemize}

\clearpage

\subsection{Pre-Requisites:}

There are three broad categories of pre-requisites that are needed:
\begin{enumerate}[topsep=-8pt]
\item Persons must be identified to fill roles within the rehearsals.
\item Services (or facsimiles) need to exist that will be used/tested throughout the 
rehearsal.
\item Elements that would not otherwise 
be available in the pre-operations LSST project will be prepared.
\end{enumerate}
 
\subsubsection{Pre-Requisites: Roles:}

Persons need to be identified that will staff various roles in this 
rehearsal.  These roles are drawn from those in operations which come
from three groups: Observatory Operations (ObsOps), Science Operations 
(SciOps), and the LSST Data Facility (LDF).
\begin{itemize}[topsep=-8pt]
\item Coordinator:  This person acts as an independent executor of the 
rehearsal.  They would be responsible for executing outside actions that drive the 
simulation (e.g., initiating a script that would start data flowing from 
summit to LDF).
\item ObsOps, Site Astronomer
\item ObsOps, Observing Specialist
\item SciOps, QA Scientist
\item SciOps, Verification and Validation Scientist
\item LDF, Operator
\item LDF, QA Scientist
\item LDF, Admin
\end{itemize}


\subsubsection{Pre-Requisites: Services \& Service Components:}

The broader pieces of the DM system that need to operate for this rehearsal are:
\begin{itemize}[topsep=-8pt]
\item Data transfer service must operate at the mountaintop for AuxTel that will send data.
The method most recently used when LATISS was on the test-stand in Tucson, rsync, is 
sufficient for this task, and can act as a backup if forwarding through the camera DAQ is 
not yet available/reliable.

\item  Nominally the long-haul networks need to be available at the time of this rehearsal.
(Note: at the time of this rehearsal we can only expect transfer rates (BASE to LDF)
of order 40 GB/s.  Therefore, ccd-scale images should require only a fraction of a second to 
transmit.)

\item A data backbone endpoint to receive and ingest incoming files must exist.

\item An automated mechanism to ingest into a Butler, on-arrival at an endpoint (OODS at BASE, DBB at LDF)
must exist.  Currently a mechanism for this exists assuming a Gen2 Butler at the LDF.

\item Processing can occur either though an automated mechanism, i.e. jobs start after data ingestion
or through a batch-production mode.  Batch production (assuming Gen2 Butler) has already been demonstrated,
if Gen3 Butler is used or on arrival processing is deemed necessary then shims may be necessary.

\item Pipeline(s) to processes the data must be in place.

\item A functional science platform (LSP) where raw and processed data products can be 
examined by staff is needed and already exists.
\end{itemize}

Exercising monitoring services for the networks, batch production, compute 
resource(s), and data backbone can be treated as a stretch goal.


\subsubsection{Pre-Requisites: Work:}\label{prework}

Work that must be completed prior to Ops Rehearsal but which is not part of 
the DM development:
\begin{itemize}[topsep=-8pt]
\item Effort to ensure that shims are available to facilitate on-arrival and/or
batch processing.  
\item Specify appropriate pipeline(s) that will be run during the rehearsal.
\item Use early AuxTel/LATISS observing to test that pipelines are appropriately 
configured for the rehearsal.
\item Make compute and storage resources are available, including a reservation for
compute resoureces to ensure timely processing can occur.
\item Create a central location (e.g. Confluence page) for the rehearsal to record 
information about incidents, problem backlog, processing and QA summaries.
\end{itemize}


\subsection{Rehearsal Outline:}
Since this proposal revolves around real observations with LATISS, some activities will
occur during the local nightime in Chile and any staff interactions that include
the Observing Specialist need to accomodate their night schedule.  This
means the communications chains among the Observing Specialist (nighttime), 
Site Astonomer (daytime), and LDF Operator will be exercised.  Currently automated
feedback from processing (at LDF) to the observing team has not been provisioned 
within the system.

A basic outline of the processes that would occur during this rehearsal are given below, 
starting from the time that afternoon calibrations are acquired on the summit.
\begin{enumerate}[topsep=-8pt]
\item \# Afternoon
\item (ObsOps) Acquire afternoon calibration.
\item (LDF) Calibrations Products Pipeline runs.
\item (SciOps) select configuration and calibrations.
\item (ALL: ObsOps+SciOps+LDF) Afternoon stand-up operations meeting.  
\item (ALL: ObsOps+SciOps+LDF) Closeout previous night's observing report
\item \# Night
\item (ObsOps+LDF) Nightly observing (data automatically transferred, ingested and processed at LDF
\item (ObsOps) create nightly reports for feedback to daytime ObsOps staff
\item \# Morning
\item (LDF) generate processing reports or discussion in stand-ups
\item (SciOps) examine input and output data from nightly observations and processing, 
\item (SciOps) generate quality reports for discussion in stand-ups
\item repeat (between 3-5 times)
\end{enumerate}

Similar to OPS Rehearsal \#1, there are "realistic" components within the outline, but 
much of focus by the actors should be on the processes, meetings, and communications.
At a minimum the actors should be considering how well this will work in day-to-day 
operations?  on weekends?  How problem are addressed?  Who gets a call and when?  
What information needs to be available between a geographically distributed team (and when)?  
Are the lines of communication between those groups sufficient?

\subsection{Assess:}

Among the activities in the Rehearsal Outline it is expected that some might influence 
the long-term development of the processes that will occur during operations but also 
might provide feedback in the DM development effort.  An example is exercising tools 
and services (e.g. the LSP) with a mind toward operational needs.  Another example is 
to inform the processes and metrics needed to make decisions about configuration and 
calibration selection in the context of both production success and production failure.

Example questions that can be asked during the assessment phase are:
\begin{itemize}[topsep=-8pt]
\item Was the rehearsal successful? How long did it take? What anomalies/failure modes were identified, and how did the team cope? 
\item What fixes are needed, and on what timescale (e.g., next ops rehearsal, or we are go for commissioning)? 
\item What improvements in procedures, documentation, frameworks, systems, and algorithms were identified and at what priority?
\item How is time and effort budgeted to plan and execute priority changes and improvements? How will the next rehearsal be planned?
\end{itemize}

%\subsection{Addendum:}
%Operations Rehearsal \#1 occurred in May 2019.  A short note, DMTN-119, gives a summary 
%report of its execution.

\clearpage



